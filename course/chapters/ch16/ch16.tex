%!TEX root = ../../e-ai_ML2.tex
%==============================================================================
\chapter{AI Transformation: From Communication to Neural Forecasting}
\label{ch:ai_transformation}
%==============================================================================

%------------------------------------------------------------------------------
\section{From Communication History to AI Assistants (1440--2022)}
\label{sec:ch16_history}
%------------------------------------------------------------------------------

The current AI transformation in weather services is best understood as part of a longer
technology trajectory. In a simplified perspective, progress is driven by two interacting
dimensions:
\begin{itemize}
\item \textbf{communication bandwidth and reach} (how information spreads),
\item \textbf{computation and interaction} (how humans can act on information).
\end{itemize}

\begin{figure}[t]
\centering
\includegraphics[width=0.62\linewidth]{images/img16/comm_history_crop.png}
\caption{Communication history and the step to LLM-era interaction.
The relevant shift is not only access to information but the ability to \emph{interact}
with information systems in natural language.}
\label{fig:ch16_comm_history}
\end{figure}

The key point is not only increasing information availability, but a shift of \emph{interaction mode}.
The web made information accessible to everyone; smartphones and social media made communication
continuous; and LLMs now add a new step: \textbf{natural language becomes an interface to knowledge and action}.

\medskip
\noindent
\textbf{1440: the printing press.}
With the printing press (Gutenberg era), information distribution changed from manual copying to
scalable replication. Knowledge became reproducible at low marginal cost and could spread reliably
across regions and generations. In the long run, this created the cultural and institutional base
for mass education, public science, and standardized documentation --- all prerequisites for later
technological transformation waves.

\medskip
\noindent
\textbf{1993: the public web transformation wave.}
The early 1990s mark the transition from local computing to global information access.
The web turned digital knowledge into a shared resource, and created the first generation
of self-service information workflows.

\medskip
\noindent
\textbf{Smartphones and social media: interaction becomes permanent.}
Smartphones and social media created permanent connectivity.
For weather services, this shift already changed user expectations:
information must be \emph{immediate, personalized, and always available}.

% --- group 1993 + smartphone to avoid whitespace and improve flow
\begin{figure}[t]
\centering

\begin{minipage}[t]{0.49\linewidth}
  \centering
  \includegraphics[width=0.92\linewidth]{images/img16/1993_computer_and_web_crop.png}
  \caption*{\footnotesize \textbf{1993: the public web.} Access to knowledge and digital workflows scales globally.}
\end{minipage}\hfill
\begin{minipage}[t]{0.49\linewidth}
  \centering
  \includegraphics[width=0.92\linewidth]{images/img16/smartphone_crop.jpeg}
  \caption*{\footnotesize \textbf{Smartphones.} Continuous interaction and highly personalized information consumption.}
\end{minipage}

\caption{From global access (web) to permanent interaction (smartphones and social media).}
\label{fig:ch16_web_and_smartphone}
\end{figure}

\medskip
\noindent
\textbf{2012--2019: ML revolution in vision and translation.}
The next ML wave was driven by machine learning successes in computer vision and machine translation.
Deep learning introduced a practical route from ``raw data'' to semantic representations.

\medskip
\noindent
\textbf{2022: chatbots cross a usability threshold.}
With chatbots, interaction itself changes.
Instead of systems requiring specialized commands and training, systems can accept
natural language inputs and iteratively refine the output in a dialogue.
For weather services this marks a qualitative shift: \textbf{language can control complex systems.}

\begin{figure}[t]
\centering

\begin{minipage}[t]{0.60\linewidth}
  \centering
  \includegraphics[width=0.94\linewidth]{images/img16/images_and_translation_crop.jpeg}
  \caption*{\footnotesize \textbf{Deep learning revolution (2012--2019).} Large improvements in vision and translation.}
\end{minipage}\hfill
\begin{minipage}[t]{0.35\linewidth}
  \centering
  \includegraphics[width=0.94\linewidth]{images/img16/chatbot2022_crop.jpeg}
  \caption*{\footnotesize \textbf{2022: Chatbots become practical.} Dialogue enables iterative refinement.}
\end{minipage}

\caption{Key milestones: deep learning breakthroughs in vision/translation and the rise of practical chatbots.}
\label{fig:ch16_translation_chatbot}
\end{figure}



%------------------------------------------------------------------------------
\begin{table}[t]
\centering
\small
\setlength{\tabcolsep}{6pt}
\renewcommand{\arraystretch}{1.15}
\begin{tabular}{p{0.32\linewidth} p{0.62\linewidth}}
\hline
\textbf{Era / technology step} & \textbf{Main change in communication \& interaction} \\
\hline
Printing press & scalable information distribution (mass replication) \\
Telegraph & long-distance messaging becomes near-instant \\
Radio & one-to-many broadcast to large audiences \\
Television & audiovisual mass media, daily public information streams \\
Computers & local digital workflows and computation for everyone \\
Internet / WWW & global access to knowledge and services \\
Social media / smartphones & permanent connectivity and personalized feeds \\
\textbf{LLMs and AI assistants} & \textbf{natural language becomes an interface to knowledge and action} \\
\hline
\end{tabular}

\caption{Simplified communication timeline and the shift towards AI assistants.
The key transition is from information access to natural-language interaction with tools and workflows.}
\label{tab:ch16_comm_timeline}
\end{table}
%------------------------------------------------------------------------------


\FloatBarrier

%------------------------------------------------------------------------------
\section{What an LLM is: tokens, attention, and next-token training}
\label{sec:ch16_llms}
%------------------------------------------------------------------------------

To understand language models, we first need a numerical representation of language.
A computer cannot operate directly on words. Instead, text is translated into \textbf{tokens},
which are integer identifiers for pieces of text (words, subwords, or even single characters).
Formally, tokenization maps a text string into a sequence
\[
\text{text} \;\longrightarrow\; (t_1,t_2,\dots,t_n),
\]
where each $t_k$ is an integer index from a finite vocabulary.

These integers are not yet meaningful by themselves.
Therefore, each token is mapped to a vector of real numbers by an \textbf{embedding}.
This embedding is the entry point into a high-dimensional \emph{representation space}:
\[
t_k \;\longrightarrow\; e(t_k)\in \mathbb{R}^d.
\]
Intuitively, the embedding vectors place tokens into a space in which
similar meanings and similar usage contexts tend to be represented by vectors
that are closer to each other (in a learned sense).

With this, language becomes a numerical sequence of vectors,
and the model can process it using linear algebra operations.

A Large Language Model (LLM) can be summarized in one sentence:

\begin{quote}
\textbf{An LLM predicts the next token in a sequence, trained on massive text data.}
\end{quote}

While this sounds simple, the emergent capabilities are substantial because:
\begin{enumerate}
\item the training corpus encodes vast amounts of human knowledge and styles,
\item the model learns representations that compress regularities of language,
\item the model performs powerful conditional generation at inference time.
\end{enumerate}

\begin{figure}[t]
\centering
\includegraphics[width=0.92\linewidth]{images/img16/transformer_all.jpeg}
\caption{Transformer-based language models: tokenization, embeddings, attention, and stacked blocks.}
\label{fig:ch16_transformer_all}
\end{figure}

\medskip
\noindent
\textbf{Tokenization and embeddings.}
Text is converted into discrete tokens $t_1, t_2, \dots, t_n$.
Each token is mapped to a vector embedding $e(t_k) \in \mathbb{R}^d$.
The model processes the sequence of embeddings:
\[
E = (e(t_1), e(t_2), \dots, e(t_n)).
\]

\medskip
\noindent
\textbf{Self-attention.}
Self-attention lets each word (or token) \emph{look at all other words} and decide which ones matter.
It computes weights that depend on the current context and then forms a weighted combination of the other tokens.

A simplified attention equation for one head is
\[
\mathrm{Attn}(Q,K,V) = \mathrm{softmax}\!\left(\frac{QK^\top}{\sqrt{d}}\right)V,
\]
where $Q,K,V$ are learned linear projections of the embeddings.
Intuitively, the dot products $QK^\top$ measure how strongly two tokens are related in the current context.
The softmax turns these scores into weights, so that important tokens contribute more to the updated representation.
The resulting attention matrix is therefore a learned, input-dependent connectivity structure.

For our course narrative, one should remember:
\begin{quote}
\textbf{Attention enables global context integration: each token can attend to relevant other tokens.}
\end{quote}

\medskip
\noindent
\textbf{Training objective: next-token prediction.}
At training, the model parameters $\theta$ are optimized to maximize the likelihood
\[
p_\theta(t_{k+1}\mid t_1,\dots,t_k)
\]
for all positions $k$ in the training data.
This ``simple'' objective is sufficient to create models that generate coherent text,
perform reasoning-like behavior, and synthesize information.

\FloatBarrier

\begin{figure}[htbp!]
\centering
\includegraphics[width=0.6\linewidth]{images/img16/rag.png}
\caption{RAG concept: retrieval from external knowledge sources complements the model’s parametric memory.}
\label{fig:ch16_rag}
\end{figure}

%------------------------------------------------------------------------------
\section{From Chat to Systems: tools, retrieval (RAG), and agents}
\label{sec:ch16_tools_rag_agents}
%------------------------------------------------------------------------------

In practice, LLMs become most valuable when they are embedded into systems.
Three extensions are essential:

\begin{itemize}
\item \textbf{Tools / function calling:} call deterministic functions and services,
\item \textbf{Retrieval (RAG):} access external documents and domain data,
\item \textbf{Memory and workflow context:} maintain state over a task.
\end{itemize}



\medskip
\noindent
\textbf{Functions vs.\ agents.}
In modern assistant architectures, a central design question is:
\begin{itemize}
\item \textbf{agent-centric:} the LLM plans, calls tools, checks results, iterates.
\item \textbf{function-centric:} the LLM selects and calls explicit functions (including agents as functions),
\end{itemize}

This is not a binary choice; practical systems combine both.
In meteorology, the tool landscape is rich: datasets, verification tools, plotting,
interpolation, model runs, documentation, and communication channels.

\medskip
\noindent
\textbf{Important: agents are tools too.}
Conceptually, an agent can be exposed to the LLM as just another function call:
\texttt{run\_agent(task, context)}.
The LLM triggers the agent, the agent executes an iterative tool loop (planning, calling tools,
verifying results), and returns a structured outcome.
In this sense, \emph{function-centric orchestration and agent-centric behavior are fully compatible}:
a single deterministic function call can launch a complex agent workflow.

\begin{figure}[htbp!]
\centering
\includegraphics[width=0.6\linewidth]{images/img16/function_or_agent_crop.png}
\caption{Assistant architecture idea: combine language-based interaction with tool usage and structured workflows.}
\label{fig:ch16_function_or_agent}
\end{figure}

\medskip
\noindent
\textbf{A weather service viewpoint.}
From a weather service perspective, the key promise is:
\begin{quote}
\textbf{Natural language becomes the universal access layer for data and tools.}
\end{quote}

This enables a new interaction model:
operators, forecasters, and developers can drive complex pipelines by dialogue,
with tool execution handling the deterministic core operations.

\FloatBarrier

%------------------------------------------------------------------------------
\section{Neural Forecasting: learning motion, fronts, and rollouts}
\label{sec:ch16_neural_forecasting}
%------------------------------------------------------------------------------

Weather prediction is fundamentally about \textbf{spatio-temporal pattern evolution}.
A central question for the AI transformation is:
\begin{quote}
\emph{How can neural networks learn to forecast dynamical evolution,
rather than only ``fit correlations''?}
\end{quote}

A minimal mental model is the advection of coherent structures.
A front, precipitation band, or convective cluster often behaves like a spatial pattern that moves and changes.
This motivates neural architectures that can represent translation-like dynamics.

\begin{figure}[t]
\centering
\includegraphics[width=0.96\linewidth]{images/img16/nn_front_template_motion.png}
\caption{Neural forecasting intuition: a ``template'' pattern plus motion and deformation.}
\label{fig:ch16_front_template}
\end{figure}

\medskip
\noindent
\textbf{The forecast mapping.}
In general, deterministic forecasting can be written as
\[
[x_{t-\Delta t}, \dots, x_t] \;\xrightarrow{\;F_\theta\;}\; x_{t+\Delta t},
\]
or in multi-step rollout form:
\[
x_{t+k\Delta t} = F_\theta^{(k)}(x_t),
\quad\text{with}\quad
F_\theta^{(k+1)}(x_t)=F_\theta(F_\theta^{(k)}(x_t)).
\]

The important operational remark:
\begin{quote}
\textbf{Long lead times are produced by repeated short-step forecasts (rollout).}
\end{quote}

\medskip
\noindent
\textbf{From nowcasting to forecasting.}
One narrative in the slides is that the boundary between nowcasting and forecasting
is increasingly bridged by AI systems.
Neural networks can learn short-step evolution from data and extend it via rollout,
sometimes supported by multi-scale architectures.

\begin{figure}[t]
\centering
\includegraphics[width=0.92\linewidth]{images/img16/nwc-to-fcst_crop.png}
\caption{From nowcasting to forecasting: short-step prediction + rollout connects the time scales.}
\label{fig:ch16_nwc_to_fcst}
\end{figure}

\medskip
\noindent
\textbf{The NWP continuum and the neural perspective.}
The ``NWP continuum'' emphasizes the broad spectrum from physics-based modeling
to data-driven prediction.
In practice, the future will contain hybrids and coexistence rather than replacement.

\begin{figure}[t]
\centering
\includegraphics[width=0.90\linewidth]{images/img16/nwp_continuum_crop.png}
\caption{NWP continuum: physics-based and ML-based methods co-evolve and interact.}
\label{fig:ch16_nwp_continuum}
\end{figure}

\FloatBarrier

%------------------------------------------------------------------------------
\section{CNN Translation Toy World: features, loss, and why it works}
\label{sec:ch16_cnn_translation}
%------------------------------------------------------------------------------

To make the learning mechanism tangible, the slides use a minimal ``toy world'':
a coherent pattern moves on a periodic domain.
The example is deliberately simple, but it captures a key mechanism:
\textbf{local feature extraction + translation equivariance}.

\medskip
\noindent
\textbf{A minimal dynamical system: signal on a circle.}
We consider a 1D periodic coordinate $x \in [0,1)$ and a signal $s(x,t)$.
At each step, the signal is shifted by some velocity:
\[
s(x,t+\Delta t) \approx s(x-u\Delta t, t),
\]
which is essentially advection on a periodic domain.

\begin{figure}[t]
\centering
\includegraphics[width=0.88\linewidth]{images/img16/signal_on_circle_3d_crop.png}
\caption{Toy world: a signal evolves on a periodic circle. Forecasting becomes learning motion and deformation.}
\label{fig:ch16_signal_circle}
\end{figure}

\medskip
\noindent
\textbf{CNN view: translation as local feature transport.}
A CNN processes local neighborhoods with shared filters.
This makes the representation naturally suited for moving structures:
the same filter detects the same feature at any location.

\begin{figure}[t]
\centering
\includegraphics[width=0.92\linewidth]{images/img16/cnn_translation_input_output.png}
\caption{CNN translation setup: learn mapping from past signal(s) to future signal.}
\label{fig:ch16_cnn_in_out}
\end{figure}

\medskip
\noindent
\textbf{Learning objective.}
Let $\hat{s}(x,t+\Delta t)=F_\theta(s(\cdot,t))$ be the prediction.
A typical loss is mean squared error:
\[
\mathcal{L}(\theta) = \frac{1}{N}\sum_{i=1}^N \| \hat{s}_i - s_i^{\mathrm{true}} \|_2^2 .
\]

Training curves typically show rapid reduction of error as the model learns
to align motion and structure.

\begin{figure}[t]
\centering
\includegraphics[width=0.72\linewidth]{images/img16/cnn_translation_loss.png}
\caption{Training loss: the CNN learns to predict the translated/deformed signal.}
\label{fig:ch16_cnn_loss}
\end{figure}

\medskip
\noindent
\textbf{Inside the CNN: hierarchical feature representations.}
The CNN does not store a ``copy'' of the signal; it learns feature representations
of increasing abstraction.
Early layers detect local gradients and bumps; deeper layers capture composite patterns.

\begin{figure}[t]
\centering
\includegraphics[width=0.96\linewidth]{images/img16/cnn_translation_features_h1.png}
\caption{CNN feature maps (example): early hidden representation.}
\label{fig:ch16_cnn_h1}
\end{figure}

\begin{figure}[t]
\centering
\includegraphics[width=0.96\linewidth]{images/img16/cnn_translation_features_h2.png}
\caption{CNN feature maps (example): intermediate hidden representation.}
\label{fig:ch16_cnn_h2}
\end{figure}

\begin{figure}[t]
\centering
\includegraphics[width=0.96\linewidth]{images/img16/cnn_translation_features_h3.png}
\caption{CNN feature maps (example): deeper hidden representation.}
\label{fig:ch16_cnn_h3}
\end{figure}

\medskip
\noindent
\textbf{Why this is not just ``correlation''.}
The toy world illustrates the general mechanism:
the network learns a representation in which the state evolution becomes predictable.
Forecasting becomes learning an operator that is consistent across space and time.

In meteorology, the same logic applies at a larger scale:
fronts, jets, clouds and precipitation structures have recognizable patterns,
and their motion is constrained by dynamics.
The neural network can exploit these regularities.

\medskip
\noindent
\textbf{Minimal code example (toy translation model).}
The following simplified code block illustrates the key learning loop
(dataset generation + CNN mapping + rollout). The actual notebook contains
a more elaborate version with diagnostics and plots.

\begin{codeonly}{python}
import numpy as np
import torch
import torch.nn as nn
import torch.nn.functional as F

# periodic 1D grid
nx = 128
x = np.linspace(0, 1, nx, endpoint=False)

def make_blob(center, width=0.06):
    dx = (x - center + 0.5) % 1.0 - 0.5
    return np.exp(-(dx**2)/(2*width**2))

def shift_periodic(signal, shift):
    return np.roll(signal, shift)

# dataset: random centers + random shifts
def sample_pair(batch=64):
    X, Y = [], []
    for _ in range(batch):
        c = np.random.rand()
        s0 = make_blob(c)
        sh = np.random.randint(1, 6)      # small motion
        s1 = shift_periodic(s0, sh)
        X.append(s0); Y.append(s1)
    X = torch.tensor(np.stack(X), dtype=torch.float32).unsqueeze(1)  # (B,1,nx)
    Y = torch.tensor(np.stack(Y), dtype=torch.float32).unsqueeze(1)
    return X, Y

class SmallCNN(nn.Module):
    def __init__(self):
        super().__init__()
        self.net = nn.Sequential(
            nn.Conv1d(1, 16, 5, padding=2),
            nn.GELU(),
            nn.Conv1d(16, 16, 5, padding=2),
            nn.GELU(),
            nn.Conv1d(16, 1, 1)
        )
    def forward(self, x):
        return self.net(x)

model = SmallCNN()
opt = torch.optim.Adam(model.parameters(), lr=1e-3)

for step in range(2000):
    Xin, Ytrue = sample_pair(batch=64)
    Yhat = model(Xin)
    loss = F.mse_loss(Yhat, Ytrue)
    opt.zero_grad(); loss.backward(); opt.step()
\end{codeonly}

\FloatBarrier

%------------------------------------------------------------------------------
\section{Weather Service Perspective: products, services, and operational ecosystems}
\label{sec:ch16_services}
%------------------------------------------------------------------------------

The AI transformation is not only about building better forecast models.
For a National Meteorological Service, a very large part of value creation happens
\textbf{after} the raw forecast is available: products, services, interpretation,
and operational integration.
In practice, weather services can easily spend \textbf{50\% or more} of their effort
on \textbf{derived products and services} --- and this share can be even higher,
because the societal value is generated through delivery, interpretation and impact.

\medskip
\noindent
\textbf{Raw forecast fields are only the beginning.}
Operational services require much more than state variables:
\begin{itemize}
\item derived variables and indicators (e.g.\ road weather indices, icing risk, wind power),
\item phenomena and event extraction (high-impact weather diagnostics),
\item warning logic and impact-based products,
\item verification, monitoring, and quality control (QC),
\item explainability, interpretation, and communication,
\item domain-specific user-facing products (aviation, hydrology, health, civil protection).
\end{itemize}
This is exactly why an AI strategy for weather services must cover the \emph{full value chain},
not only the forecasting model.

\medskip
\noindent
\textbf{Operational AI forecasting is already real.}
The slides emphasize an important reality check:
AI forecasting in Europe is no longer ``research only''.
Several systems are already in production use (e.g.\ AIFS, AICON, BRIS),
and they complement classical NWP by providing:
\begin{itemize}
\item fast state-to-state forecasts (minutes instead of hours),
\item competitive large-scale skill for key variables,
\item robust baselines and rapid experimentation.
\end{itemize}

\begin{figure}[t]
\centering
\includegraphics[width=0.30\linewidth]{images/img16/already1.png}\hfill
\includegraphics[width=0.22\linewidth]{images/img16/already2.png}\hfill
\includegraphics[width=0.30\linewidth]{images/img16/already3.png}
\caption{Operational AI forecasting is already in production use in Europe (slide narrative:
``not research only''). This strengthens the case for systematic integration into the
weather service toolbox.}
\label{fig:ch16_already_operational}
\end{figure}

\medskip
\noindent
\textbf{Ecosystem viewpoint: a portfolio of complementary libraries.}
The European ML-for-NWP ecosystem develops along several complementary lines.
A key slide message is that this is \textbf{good news}:
no single framework needs to do everything; modular building blocks enable rapid innovation,
and shared standards support interoperability and reuse.
In particular, the ecosystem includes complementary strengths such as:
\begin{itemize}
\item \textbf{Anemoi:} end-to-end AI weather models (training + inference stack),
\item \textbf{mfai:} transformers / vision methods and many applications,
\item \textbf{MLCast:} observation-driven nowcasting components,
\item \textbf{FRAIM:} products and services across the full AI/ML value chain.
\end{itemize}

\begin{figure}[t]
\centering
\includegraphics[width=0.92\linewidth]{images/img16/library_ecosystem2.png}
\caption{Library ecosystem: complementary strengths, one shared mission.
The core point is modularity and interoperability rather than a single monolith.}
\label{fig:ch16_library_ecosystem2}
\end{figure}

\medskip
\noindent
\textbf{What we can do already: full value chain competence.}
We already have strong building blocks in place,
spanning both core forecasting and applied product development:
\begin{itemize}
\item an operational framework for AI-based forecasting (Anemoi),
\item product and service libraries (mfai, FRAIM) for:
\begin{itemize}
\item downscaling and high-resolution products,
\item road weather services,
\item high-impact weather feature extraction,
\item interpretation and explainability,
\item nowcasting of observation fields (radiation, precipitation, \dots).
\end{itemize}
\end{itemize}
This is not just technical capability: it is the \textbf{organizational foundation}
for sustainable AI operations, collaboration, and benchmarking.

\medskip
\noindent
\textbf{Physics and AI: why physics remains essential.}
A central warning in the slides is that weather is not just a ``pattern problem''.
Forecasting is consistent evolution of a physical state:
conservation laws, balances, constraints, multi-scale coupling.
Pure AI models can show strong short-term skill but still drift long-term when small violations
accumulate (moisture, energy, mass), and rare extremes require physical consistency.
The long-term direction is therefore clear:
\begin{quote}
\textbf{Future AI forecast systems will be ``Physics + AI''.}
\end{quote}

\begin{figure}[t]
\centering
\includegraphics[width=0.70\linewidth]{images/img16/earth_forecast2.png}
\caption{Physics and AI: AI gives speed and learning; physics gives truth and trust.
The operational goal is reliable evolution under constraints, not only local pattern accuracy.}
\label{fig:ch16_physics_ai_earth}
\end{figure}

\medskip
\noindent
\textbf{Key message for weather services.}
The AI transformation becomes truly valuable when forecasting engines are connected
to product/service ecosystems and standards-based operational integration.
In other words:
\begin{quote}
\textbf{Forecast engines + product/service ecosystems + operational platforms.}
\end{quote}

\FloatBarrier


%------------------------------------------------------------------------------
\section{Outlook: transformation of workflows and service building}
\label{sec:ch16_outlook}
%------------------------------------------------------------------------------

The final part of the lecture is pragmatic and action-oriented.
AI will not only provide new forecast engines;
it will reshape workflows, software architectures, and how we build services.
The slide narrative highlights three connected points: \textbf{tools},
\textbf{standards}, and \textbf{workflow redesign}.

\medskip
\noindent
\textbf{Tool calling becomes a strategic capability.}
Modern AI assistants become powerful when they can call trusted tools.
This is currently a fast-moving technology race:
\begin{itemize}
\item OpenAI: tool calling with JSON schema and structured outputs,
\item Anthropic: MCP (Model Context Protocol) as connector architecture,
\item Google: Gemini function calling within its ecosystem,
\item Linux Foundation / AAIF: interoperability to avoid vendor lock-in.
\end{itemize}
The key operational message for weather services is:
\begin{quote}
\textbf{LLMs need good tool definitions --- and we can use this to build better services.}
\end{quote}

\begin{figure}[t]
\centering
\includegraphics[width=0.22\linewidth]{images/img16/race01.png}\hfill
\includegraphics[width=0.22\linewidth]{images/img16/race02.png}\hfill
\includegraphics[width=0.22\linewidth]{images/img16/race03.png}\hfill
\includegraphics[width=0.22\linewidth]{images/img16/race04.png}
\caption{Tool calling and agent interoperability: multiple technology streams push towards
standardized access to tools and enterprise systems.
For weather services this translates into a concrete opportunity:
expose operational expertise as callable tools.}
\label{fig:ch16_tool_calling_race}
\end{figure}

\medskip
\noindent
\textbf{DAWID opportunity: our trusted functions at our fingertips.}
A central slide message is that a platform such as DAWID can expose existing operational
capabilities as tools with clean interfaces:
\begin{itemize}
\item tools are trusted DWD / ECMWF / NMS functions (not generic chat),
\item each tool encodes a workflow step we already know,
\item the LLM becomes orchestrator and user interface.
\end{itemize}
This yields immediate impact: faster exploration, reproducible results,
and knowledge transfer (tools carry expertise).

\begin{figure}[t]
\centering
\includegraphics[width=0.48\linewidth]{images/img16/fraim01.png}\hfill
\includegraphics[width=0.48\linewidth]{images/img16/fraim02.png}
\caption{DAWID and FRAIM in the slide narrative: make operational capabilities accessible
via tool interfaces --- ``our functions'' become reusable building blocks.}
\label{fig:ch16_dawid_fraim}
\end{figure}

\medskip
\noindent
\textbf{Workflow redesign: if a task repeats, it should become a tool.}
One of the strongest action messages is: rebuild services by analysing daily work,
finding waste, and turning repeated tasks into tools with clear I/O interfaces.
This can reduce copy/paste workflows, waiting time, and loss of context across files,
emails and chats.
In short:
\begin{quote}
\textbf{If a task repeats, it should become a tool.}
\end{quote}

\medskip
\noindent
\textbf{Bring your domain expertise: make AI your own.}
LLMs and neural networks do not automatically understand meteorology:
processes, states, balances, physical limits, plausibility, regimes, and extremes.
Operational quality requires embedding domain knowledge into:
\begin{itemize}
\item variables, events, diagnostics, and what matters,
\item physical constraints and what is allowed,
\item product requirements and what is useful.
\end{itemize}
This is the human core contribution in the AI era: not outsourcing expertise,
but encoding it into tools, workflows and constraints.

\begin{figure}[t]
\centering

\begin{minipage}[t]{0.48\linewidth}
  \centering
  \vspace{0pt}
  \includegraphics[width=\linewidth,height=0.60\textheight,keepaspectratio]{images/img16/physics_and_ai.png}

  \vspace{1mm}
  {\footnotesize \textbf{Domain expertise.} AI becomes powerful when grounded in meteorology.}
\end{minipage}
\hfill
\begin{minipage}[t]{0.48\linewidth}
  \centering
  \vspace{0pt}
  \includegraphics[width=\linewidth,height=0.60\textheight,keepaspectratio]{images/img16/ai_and_weather3.png}

  \vspace{1mm}
  {\footnotesize \textbf{System transformation.} Models, tools, workflows and products evolve together.}
\end{minipage}

\vspace{1mm}
\caption{Key messages of the AI transformation for weather services:
domain expertise is essential for trust, and operational impact requires a full-system approach.}
\label{fig:ch16_outlook_pair}
\end{figure}



\medskip
\noindent
\textbf{Outlook.}
The transformation is therefore not only a model question, but a system question:
AI-enabled weather services will be characterized by
\begin{itemize}
\item language-driven access to complex toolchains,
\item automation of analysis, summarization and reporting,
\item rapid prototyping of products and service variants,
\item continuous evaluation and improvement cycles.
\end{itemize}


