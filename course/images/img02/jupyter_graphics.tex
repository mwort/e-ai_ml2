\documentclass[tikz,border=10pt]{standalone}
\usepackage{tikz}
\usetikzlibrary{shapes.geometric, arrows.meta, positioning, shadows, fit}

\begin{document}

\tikzset{
    box/.style={
        rectangle,
        rounded corners,
        minimum width=3cm,
        minimum height=1.2cm,
        text centered,
        draw=black,
        fill=cyan!30,
        drop shadow
    },
    server/.style={
        rectangle,
        rounded corners,
        minimum width=3.5cm,
        minimum height=1.5cm,
        text centered,
        draw=black,
        fill=orange!30,
        drop shadow
    },
    storage/.style={
        rectangle,
        rounded corners,
        minimum width=3cm,
        minimum height=1.2cm,
        text centered,
        draw=black,
        fill=purple!20,
        drop shadow
    },
    cloud/.style={
        ellipse,
        minimum width=2.5cm,
        minimum height=1cm,
        text centered,
        draw=black,
        fill=blue!15,
        drop shadow
    },
    arrow/.style={
        thick,
        -Stealth,
        color=blue!70
    },
    bigarrow/.style={
        very thick,
        -Stealth,
        color=teal!70
    }
}

\begin{tikzpicture}[node distance=2.5cm]

% Local Machine
\node[box] (browser) {Web Browser\\{\small (Chrome, Firefox)}};
\node[below=0.5cm of browser, font=\small\itshape] (local) {Local Machine};
\node[fit={(browser) (local)}, draw, dashed, rounded corners, line width=1pt, color=cyan!60, label=above:{\textbf{Client Side}}, inner sep=0.6cm] (client-box) {};

% Network
\node[cloud, right=3cm of browser] (network) {Internet/\\SSH Tunnel};

% Remote Server
\node[server, right=3cm of network] (jupyter) {Jupyter Server\\{\small Port 8888}};
\node[box, above=1.5cm of jupyter] (kernel) {Python Kernel\\{\small Computation}};
\node[storage, below=1.5cm of jupyter] (files) {Files \& Notebooks\\{\small .ipynb}};
\node[below=0.5cm of files, font=\small\itshape] (remote) {Remote Server (HPC, Cloud, VM)};
\node[fit={(jupyter) (kernel) (files) (remote)}, draw, dashed, rounded corners, line width=1pt, color=orange!60, label=above:{\textbf{Server Side}}, inner sep=0.6cm] (server-box) {};

% Connections
\draw[bigarrow, <->] (browser) -- node[above] {\small HTTP/WebSocket} (network);
\draw[bigarrow, <->] (network) -- node[above] {\small SSH Tunnel} node[below] {\small localhost:8888} (jupyter);

% Better arrow routing on the right side
\draw[arrow, <->] (jupyter.north) -- (kernel.south);
\draw[arrow, <->] (jupyter.south) -- (files.north);

% Labels
\node[below=0.3cm of client-box, text width=6cm, align=center, font=\footnotesize] {
    Access via: \texttt{localhost:9001}
};

\node[below=0.3cm of network, text width=7cm, align=center, font=\footnotesize] {
    SSH command: \\\texttt{ssh -L 9001:localhost:8888 user@server}
};

\end{tikzpicture}

\end{document}